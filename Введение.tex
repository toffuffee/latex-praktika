\section*{ВВЕДЕНИЕ}
\addcontentsline{toc}{section}{ВВЕДЕНИЕ}

Современные предприятия сталкиваются с необходимостью повышения эффективности внутренних процессов, улучшения коммуникации между сотрудниками и централизации бизнес-сервисов. В условиях цифровой трансформации особую актуальность приобретает автоматизация управления персоналом — ключевого ресурса любой организации.

Традиционно компании используют множество разрозненных решений: почтовые клиенты, мессенджеры, системы управления задачами, календари, сервисы видеосвязи и облачные хранилища. Отсутствие единого интерфейса и централизованного подхода к управлению такими сервисами приводит к фрагментированности данных, росту временных затрат и снижению продуктивности сотрудников.

Для решения указанных проблем в данной работе разрабатывается универсальная веб-платформа, объединяющая основные модули для эффективной организации внутренних коммуникаций и управления задачами внутри компании. Платформа реализована с использованием современных веб-технологий, таких как TypeScript, React, JMAP, и представляет собой прогрессивное веб-приложение (PWA) с модульной архитектурой.

Разработка охватывает создание сервисов: электронной почты, календаря, контактов, файлового хранилища, видеоконференций, чатов, панели управления и системы управления проектами. Все компоненты взаимосвязаны и доступны через единый пользовательский интерфейс.

\emph{Цель настоящей работы} — разработка и внедрение веб-платформы, обеспечивающей автоматизацию бизнес-процессов управления персоналом, повышение прозрачности взаимодействия и улучшение координации в рамках одного цифрового пространства.

Для достижения поставленной цели необходимо решить следующие \emph{задачи:}
\begin{itemize}
\item провести анализ предметной области и существующих решений;
\item разработать концептуальную модель веб-платформы;
\item спроектировать архитектуру платформы и пользовательский интерфейс;
\item реализовать модули платформы с применением современных технологий;
\item провести тестирование готовой системы и внедрить её в организацию.
\end{itemize}

\emph{Структура и объём работы.} Отчет состоит из введения, четырёх разделов основной части, заключения, списка использованных источников и двух приложений. Общий объем текста составляет \formbytotal{lastpage}{страниц}{у}{ы}{}. Работа содержит графический материал (приложение А) и исходный код (приложение Б).

\emph{Во введении} обозначены цель, задачи и структура работы.

\emph{Первый раздел} посвящён анализу предметной области и описанию проблем, решаемых разрабатываемой системой.

\emph{Второй раздел} содержит техническое задание: формулируются требования к функциональности платформы, интерфейсу и архитектуре.

\emph{Третий раздел} включает проектные решения: выбор технологий, диаграммы компонентов, структура базы данных и описание взаимодействий.

\emph{Четвёртый раздел} содержит описание реализации, модульного и системного тестирования платформы.

В заключении подведены итоги работы и сделаны выводы по результатам внедрения платформы.

В приложении А представлен графический материал.
В приложении Б представлены фрагменты исходного кода. 
