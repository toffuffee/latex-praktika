\section{Анализ предметной области}
\subsection{Характеристика предприятия и его деятельности}

Компания \textbf{ВТИ-Сервис} — ведущая ИТ-компания Курской области, аккредитованная Министерством цифрового развития РФ. Она занимается поставкой программного обеспечения и оборудования для автоматизации торговли, сферы услуг и промышленности, а также разработкой собственных решений. Основной целью компании является комплексная автоматизация бизнес-процессов на предприятиях различного масштаба.

Основные направления деятельности включают:

\begin{enumerate}
  \item Компания осуществляет разработку и внедрение программного обеспечения. Решения адаптированы под различные отрасли и включают:
  \begin{itemize}
    \item автоматизацию торговли;
    \item автоматизацию общественного питания;
    \item управление складским учетом.
  \end{itemize}

  \item Осуществляется поставка торгового оборудования, обеспечивающего полный цикл обслуживания клиента. В него входят:
  \begin{itemize}
    \item онлайн-кассы;
    \item сканеры штрих-кодов;
    \item принтеры чеков и этикеток;
    \item весовое оборудование.
  \end{itemize}

  \item Компания предоставляет ИТ-услуги, направленные на поддержку и развитие инфраструктуры клиентов. Среди них:
  \begin{itemize}
    \item ИТ-аутсорсинг;
    \item внедрение и сопровождение продуктов 1С;
    \item техническое обслуживание и настройка оборудования.
  \end{itemize}

  \item В области безопасности внедряются современные решения. Компания проектирует и устанавливает:
  \begin{itemize}
    \item системы видеонаблюдения;
    \item системы контроля и управления доступом;
    \item средства удалённого мониторинга.
  \end{itemize}

  \item ВТИ-Сервис активно разрабатывает собственное программное обеспечение, включая мобильные приложения. Например, приложение \textit{DM.ТОИР} используется для технического обслуживания и ремонта основных фондов предприятий.

  \item Все решения компании соответствуют действующему законодательству. Организация обеспечивает:
  \begin{itemize}
    \item передачу фискальных данных в налоговые органы;
    \item соответствие требованиям обязательной маркировки товаров.
  \end{itemize}
\end{enumerate}

\subsection{Постановка задачи}

Для оптимизации внутренних процессов предприятия разрабатывается веб-платформа, направленная на автоматизацию управления персоналом. Основные проблемы, решаемые системой:
\begin{itemize}
  \item фрагментированность бизнес-сервисов;
  \item неэффективная коммуникация между сотрудниками;
  \item отсутствие единой системы хранения информации.
\end{itemize}

\subsection{Аналогичные решения и их анализ}

Среди известных решений на рынке выделяются:

Bitrix24
\begin{itemize}
  \item \textbf{Функции:} CRM, управление проектами, задачи, календари, мессенджер, телефония, документы.
  \item \textbf{Достоинства:} множество модулей в одном интерфейсе, интеграции с 1С и внешними API, развитый документооборот.
  \item \textbf{Недостатки:} высокая стоимость при масштабировании, громоздкий интерфейс, избыточность для небольших компаний.
\end{itemize}

Jira + Confluence + Zoom
\begin{itemize}
  \item \textbf{Функции:} трекинг задач, документация, видеоконференции.
  \item \textbf{Достоинства:} мощные инструменты для командной работы, поддержка agile-методологий.
  \item \textbf{Недостатки:} высокая кривая обучения, англоязычный интерфейс по умолчанию, дорогая лицензия.
\end{itemize}

\subsection{Преимущества разрабатываемой системы}

Разрабатываемая веб-платформа предлагает:
\begin{itemize}
  \item Единую точку входа к сервисам: почта, файлы, календарь, проекты, мессенджер, ВКС и др.;
  \item Централизованное управление пользователями и задачами;
  \item Интуитивно понятный, адаптивный интерфейс;
  \item Возможность расширения и интеграции с другими системами;
  \item Открытую архитектуру для масштабирования и кастомизации.
\end{itemize}

\subsection{Вывод}

Анализ предметной области и имеющихся решений показал, что разрабатываемая система будет актуальна для малого и среднего бизнеса, стремящегося к цифровой трансформации. Использование модульной архитектуры и современных технологий обеспечит гибкость, удобство и функциональную полноту, превосходящую отдельные существующие решения.