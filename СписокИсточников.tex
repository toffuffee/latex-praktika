\addcontentsline{toc}{section}{СПИСОК ИСПОЛЬЗОВАННЫХ ИСТОЧНИКОВ}

\begin{thebibliography}{9}

    \bibitem{python1} Лутц, М. Изучаем Python / М. Лутц. – 5-е изд. – Санкт-Петербург: Питер, 2021. – 1300 с. – ISBN 978-5-4461-1556-3. – Текст~: непосредственный.

    \bibitem{python2} Свигарт, А. Автоматизация рутинных задач с помощью Python / А. Свигарт. – 2-е изд. – Санкт-Петербург: Питер, 2021. – 592 с. – ISBN 978-5-4461-1865-6. – Текст~: непосредственный.

    \bibitem{python3} Матей, Н. Python. К вершинам мастерства / Н. Матей. – Санкт-Петербург: Питер, 2022. – 432 с. – ISBN 978-5-4461-0926-5. – Текст~: непосредственный.

    \bibitem{react1} React – официальная документация [Электронный ресурс]. – Режим доступа: https://react.dev/ (дата обращения: 07.05.2025). – Текст~: электронный.

    \bibitem{javascript1} Дукетт, Дж. JavaScript и jQuery. Интерактивная разработка веб-интерфейсов / Дж. Дукетт. – Санкт-Петербург: Питер, 2016. – 640 с. – ISBN 978-5-496-01749-7. – Текст~: непосредственный.

    \bibitem{javascript2} Флэнаган, Д. JavaScript. Подробное руководство / Д. Флэнаган. – Москва: Вильямс, 2020. – 960 с. – ISBN 978-5-8459-1904-5. – Текст~: непосредственный.

    \bibitem{js3} Марейн, М. Выразительный JavaScript / М. Марейн. – Санкт-Петербург: Питер, 2020. – 472 с. – ISBN 978-5-4461-1188-6. – Текст~: непосредственный.

    \bibitem{css1} Веру, Л. CSS-секреты. 47 советов по улучшению веб-интерфейсов / Л. Веру. – Санкт-Петербург: Питер, 2017. – 368 с. – ISBN 978-5-496-02699-4. – Текст~: непосредственный.

    \bibitem{css2} Юэнс, В. CSS. Карманный справочник / В. Юэнс. – Москва: Символ-Плюс, 2020. – 272 с. – ISBN 978-5-93286-362-0. – Текст~: непосредственный.

    \bibitem{htmlcss} Дэкетт, Д. HTML и CSS. Разработка и создание веб-сайтов / Д. Дэкетт. – Москва: Эксмо, 2014. – 480 с. – ISBN 978-5-699-64193-2. – Текст~: непосредственный.

\end{thebibliography}
