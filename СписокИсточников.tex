\addcontentsline{toc}{section}{СПИСОК ИСПОЛЬЗОВАННЫХ ИСТОЧНИКОВ}

\begin{thebibliography}{9}

    \bibitem{react3} Бэнкс А., Порселло Е. React и Redux. Функциональная веб-разработка. — СПб.: Питер, 2018. — 336 с. — ISBN 978-5-4461-0669-1. — Текст: непосредственный.

    \bibitem{htmlcss} Дэкетт Д. HTML и CSS. Разработка и создание веб-сайтов. — М.: Эксмо, 2014. — 480 с. — ISBN 978-5-699-64193-2. — Текст: непосредственный.

    \bibitem{javascript1} Флэнаган Д. JavaScript. Подробное руководство. — М.: Вильямс, 2020. — 960 с. — ISBN 978-5-8459-1904-5. — Текст: непосредственный.

    \bibitem{python3} Матей Н. Python. К вершинам мастерства. — СПб.: Питер, 2022. — 432 с. — ISBN 978-5-4461-0926-5. — Текст: непосредственный.

    \bibitem{javascript2} Марейн М. Выразительный JavaScript. — СПб.: Питер, 2020. — 472 с. — ISBN 978-5-4461-1188-6. — Текст: непосредственный.

    \bibitem{python1} Лутц М. Изучаем Python. — 5-е изд. — СПб.: Питер, 2021. — 1300 с. — ISBN 978-5-4461-1556-3. — Текст: непосредственный.

    \bibitem{python2} Свигарт А. Автоматизация рутинных задач с помощью Python. — 2-е изд. — СПб.: Питер, 2021. — 592 с. — ISBN 978-5-4461-1865-6. — Текст: непосредственный.

    \bibitem{react2} Вайс А. React в действии. — СПб.: Питер, 2019. — 384 с. — ISBN 978-5-4461-1091-9. — Текст: непосредственный.

    \bibitem{css1} Веру Л. CSS-секреты. 47 советов по улучшению веб-интерфейсов. — СПб.: Питер, 2017. — 368 с. — ISBN 978-5-496-02699-4. — Текст: непосредственный.

    \bibitem{css2} Юэнс В. CSS. Карманный справочник. — М.: Символ-Плюс, 2020. — 272 с. — ISBN 978-5-93286-362-0. — Текст: непосредственный.

\end{thebibliography}